\chapter{Conclusiones}
\label{conclusiones}

Este capítulo presenta las conclusiones derivadas del desarrollo e implementación del sistema de aprendizaje de idiomas basado en técnicas de \gls{rl} y arquitecturas \gls{transformers}. Se analizan los logros alcanzados, las contribuciones realizadas, las limitaciones identificadas y las líneas futuras de investigación y desarrollo.

\section{Logros del Proyecto}
\label{sec:logros-proyecto}

El presente trabajo ha logrado desarrollar un sistema integral de aprendizaje de idiomas que cumple con los objetivos planteados inicialmente:

\begin{itemize}
    \item \textbf{Personalización adaptativa}: Se ha implementado con éxito un sistema basado en \gls{ppo} que optimiza la ruta de aprendizaje según el perfil y progreso individual del estudiante, ajustando dinámicamente la dificultad del contenido.
    
    \item \textbf{Interacción conversacional avanzada}: La integración de modelos \gls{llm} y un sistema \gls{rag} ha permitido generar diálogos contextualizados y naturales, proporcionando una experiencia conversacional realista.
    
    \item \textbf{Desarrollo de habilidades lingüísticas completas}: El sistema integra con éxito tecnologías de procesamiento de voz (\gls{tts} y \gls{stt}) para el desarrollo simultáneo de habilidades de comprensión auditiva y producción oral.
    
    \item \textbf{Gestión eficiente del conocimiento}: La implementación del sistema \gls{rag} proporciona acceso contextualizado a recursos educativos relevantes, mejorando la precisión y relevancia de las respuestas del sistema.
    
    \item \textbf{Arquitectura modular y extensible}: El diseño del sistema permite su evolución y adaptación a nuevos requisitos, facilitando la incorporación de mejoras y nuevas funcionalidades.
\end{itemize}

\section{Contribuciones y Aportes}
\label{sec:contribuciones}

Las principales contribuciones de este trabajo al campo del aprendizaje de idiomas asistido por \gls{ia} son:

\subsection{Avances Técnicos}
\label{subsec:avances-tecnicos}

\begin{itemize}
    \item \textbf{Modelo PPO optimizado para educación}: Se ha desarrollado un modelo \gls{ppo} específicamente adaptado al contexto educativo, capaz de tomar decisiones pedagógicas informadas basadas en múltiples métricas de rendimiento.
    
    \item \textbf{Integración efectiva de LLM y RAG}: El sistema demuestra una implementación eficiente de la combinación de modelos de lenguaje de gran escala con recuperación aumentada de generación, proporcionando respuestas contextualmente relevantes y educativamente significativas.
    
    \item \textbf{Pipeline de procesamiento de voz optimizado}: La adaptación de Faster-Whisper y Kokoro-TTS para el contexto educativo representa una optimización significativa en términos de eficiencia y precisión para aplicaciones de aprendizaje de idiomas.
\end{itemize}

\subsection{Aportes Metodológicos}
\label{subsec:aportes-metodologicos}

\begin{itemize}
    \item \textbf{Framework de evaluación multidimensional}: Se ha desarrollado un enfoque sistemático para evaluar tanto el rendimiento técnico del sistema como su impacto educativo real.
    
    \item \textbf{Generación de escenarios representativos}: La metodología desarrollada para crear y evaluar escenarios de aprendizaje representativos proporciona un marco útil para futuras investigaciones en sistemas adaptativos.
    
    \item \textbf{Diseño centrado en el estudiante}: El proyecto ha implementado un enfoque que prioriza la experiencia del estudiante, adaptando las tecnologías a las necesidades pedagógicas reales.
\end{itemize}

\section{Limitaciones del Trabajo}
\label{sec:limitaciones}

A pesar de los logros alcanzados, es importante reconocer las limitaciones actuales del sistema:

\begin{itemize}
    \item \textbf{Evaluación preliminar}: Las pruebas realizadas, aunque prometedoras, se han limitado a un grupo reducido de usuarios en un entorno controlado. Se requieren estudios más extensos y longitudinales para validar completamente la eficacia del sistema.
    
    \item \textbf{Cobertura lingüística}: Aunque el sistema soporta múltiples idiomas, la calidad y profundidad de los recursos educativos varían significativamente entre ellos, con mayor robustez en idiomas mayoritarios como el inglés y el español.
    
    \item \textbf{Dependencia de recursos computacionales}: El sistema actual requiere recursos computacionales considerables, lo que puede limitar su accesibilidad en entornos con restricciones tecnológicas.
    
    \item \textbf{Aspectos culturales del lenguaje}: El sistema muestra limitaciones en la comprensión y generación de aspectos culturalmente específicos del lenguaje, como modismos, humor o referencias culturales locales.
\end{itemize}

\section{Líneas Futuras}
\label{sec:lineas-futuras}

Este trabajo abre diversas líneas de investigación y desarrollo futuro:

\subsection{Mejoras Técnicas a Corto Plazo}
\label{subsec:mejoras-corto-plazo}

\begin{itemize}
    \item \textbf{Ampliación de la base de conocimientos}: Expandir y enriquecer la base de conocimientos del sistema \gls{rag}, incorporando recursos educativos más diversos y actualizados.
    
    \item \textbf{Mejora del sistema de corrección}: Implementar técnicas más sofisticadas para la detección y corrección de errores lingüísticos en tiempo real.
\end{itemize}

\subsection{Visión a Largo Plazo}
\label{subsec:vision-largo-plazo}

\begin{itemize}
    \item \textbf{Sistemas multimodales}: Integrar comprensión y generación multimodal (texto, voz, gestos, expresiones faciales) para una experiencia de aprendizaje más inmersiva y completa.
    
    \item \textbf{Adaptación a contextos específicos}: Desarrollar versiones especializadas del sistema para contextos educativos específicos, como la enseñanza de idiomas para fines específicos (turismo, negocios, medicina, etc.).
    
    \item \textbf{Aprendizaje colaborativo}: Explorar la integración de sistemas de aprendizaje colaborativo que fomenten la interacción entre estudiantes y la co-creación de conocimiento.
\end{itemize}

\section{Reflexiones Finales}
\label{sec:reflexiones-finales}

El desarrollo de este sistema representa un paso significativo hacia la personalización efectiva del aprendizaje de idiomas mediante tecnologías de \gls{ia}. La combinación de \gls{rl}, arquitecturas \gls{transformers} y sistemas \gls{rag} demuestra el potencial de las tecnologías actuales para transformar fundamentalmente el campo educativo.

El valor principal del sistema no reside únicamente en sus capacidades técnicas, sino en su potencial para democratizar el acceso a experiencias de aprendizaje personalizadas y efectivas. La adaptación dinámica a las necesidades individuales permite superar las limitaciones de los enfoques tradicionales, que a menudo fallan en proporcionar el apoyo específico que cada estudiante requiere.

Sin embargo, es importante reconocer que la tecnología, por muy avanzada que sea, representa solo una herramienta al servicio de objetivos pedagógicos más amplios. El sistema desarrollado no pretende reemplazar a los educadores humanos, sino complementar su labor, proporcionando un entorno de práctica y retroalimentación constante que enriquece la experiencia educativa global.

La verdadera medida del éxito de este proyecto será su capacidad para facilitar el aprendizaje de idiomas de manera más eficiente, inclusiva y motivadora, contribuyendo así a derribar las barreras lingüísticas que separan a las personas y comunidades en un mundo cada vez más interconectado.

Como reflexión final, cabe destacar que el campo de la \gls{ia} aplicada a la educación se encuentra en constante evolución, y este trabajo representa solo un punto de partida para desarrollos futuros que continuarán transformando la manera en que aprendemos y enseñamos idiomas.