% Glosario

% Estado del arte
\newglossaryentry{rl}{
  name={RL},
  description={Reinforcement Learning - Aprendizaje por Refuerzo, una rama de la IA que permite a los sistemas aprender a través de la interacción con un entorno},
  first={Reinforcement Learning (RL)\glsadd{rl}}
}

\newglossaryentry{multi-agent}{
  name={Sistema Multi-Agente},
  description={Sistema compuesto por múltiples agentes inteligentes que interactúan entre sí para resolver problemas complejos},
  first={Sistema Multi-Agente\glsadd{multi-agent}}
}

\newglossaryentry{data-mining}{
  name={Data Mining},
  description={Proceso de descubrir patrones y relaciones en grandes conjuntos de datos},
  first={Data Mining\glsadd{data-mining}}
}

\newglossaryentry{agentic-ia}{
  name={IA Agéntica},
  description={Un tipo de inteligencia artificial que actúa de manera autónoma y proactiva, tomando decisiones y realizando acciones para alcanzar objetivos específicos}
}

\newglossaryentry{open-source}{
  name={Código Abierto},
  description={Software cuyo código fuente está disponible públicamente y puede ser modificado y distribuido por cualquier persona}
}

\newglossaryentry{sla}{
  name={SLA},
  description={Second Language Acquisition - Proceso de adquisición de una segunda lengua},
  first={Second Language Acquisition (SLA)\glsadd{sla}}
}

\newglossaryentry{input}{
  name={Input Hypothesis},
  description={Hipótesis propuesta por Krashen que establece que los aprendices progresan cuando reciben input comprensible},
  first={Input Hypothesis\glsadd{input}}
}

\newglossaryentry{ia}{
  name={IA},
  description={Inteligencia Artificial - Conjunto de tecnologías que permiten a las máquinas aprender, razonar y tomar decisiones},
  first={Inteligencia Artificial (IA)\glsadd{ia}}
}

\newglossaryentry{transformers}{
  name={Transformers},
  description={Arquitectura de red neuronal que ha revolucionado el procesamiento del lenguaje natural},
  first={Transformers\glsadd{transformers}}
}

\newglossaryentry{tts}{
  name=TTS,
  description={Text-to-Speech (Texto a Voz). Sistema que convierte texto escrito en habla sintetizada},
  first={Text-to-Speech (TTS)\glsadd{tts}},
  long={Text-to-Speech}
}

\newglossaryentry{stt}{
  name=STT,
  description={Speech-to-Text (Voz a Texto). Sistema que convierte el habla en texto escrito mediante reconocimiento automático del habla},
  first={Speech-to-Text (STT)\glsadd{stt}},
  long={Speech-to-Text}
}

\newglossaryentry{rag}{
  name={RAG},
  description={Retrieval-Augmented Generation - Sistema que combina la recuperación de información con la generación de texto},
  first={Retrieval-Augmented Generation (RAG)\glsadd{rag}}
}

\newglossaryentry{llm}{
  name={LLM},
  description={Large Language Model - Modelo de lenguaje de gran escala},
  first={Large Language Model (LLM)\glsadd{llm}}
}

\newglossaryentry{nlp}{
  name={NLP},
  description={Natural Language Processing - Procesamiento del Lenguaje Natural},
  first={Natural Language Processing (NLP)\glsadd{nlp}}
}

\newglossaryentry{tensorflow}{
  name={TensorFlow},
  description={Biblioteca de código abierto para aprendizaje automático desarrollada por Google},
  first={TensorFlow\glsadd{tensorflow}}
}

\newglossaryentry{py-torch}{
  name={PyTorch},
  description={Biblioteca de aprendizaje profundo de código abierto desarrollada por Meta},
  first={PyTorch\glsadd{py-torch}}
}

% Marco teórico
\newglossaryentry{its}{
  name={ITS},
  description={Intelligent Tutoring System - Sistema de Tutoría Inteligente},
  first={Intelligent Tutoring System (ITS)\glsadd{its}}
}

\newglossaryentry{adaptive}{
  name={Sistema Adaptativo},
  description={Sistema capaz de modificar su comportamiento según las necesidades del usuario},
  first={Sistema Adaptativo\glsadd{adaptive}}
}

\newglossaryentry{recommender}{
  name={Sistema de Recomendación},
  description={Sistema que sugiere contenido relevante basado en el perfil del usuario y su comportamiento},
  first={Sistema de Recomendación\glsadd{recommender}}
}

\newglossaryentry{attention}{
  name={Mecanismo de Atención},
  description={Componente clave de la arquitectura Transformer que permite al modelo enfocarse en diferentes partes de la entrada según su relevancia},
  first={Mecanismo de Atención\glsadd{attention}}
}

\newglossaryentry{feed-forward}{
  name={Feed-Forward},
  description={Capa de red neuronal que aplica una transformación lineal seguida de una función de activación},
  first={Feed-Forward\glsadd{feed-forward}}
}

\newglossaryentry{self-attention}{
  name={Auto-Atención},
  description={Mecanismo que permite a un modelo evaluar las relaciones entre todas las posiciones de una secuencia},
  first={Auto-Atención\glsadd{self-attention}}
}

\newglossaryentry{token}{
  name={Token},
  description={Unidad básica de procesamiento en modelos de lenguaje, que puede ser una palabra, subpalabra o carácter},
  first={Token\glsadd{token}}
}

\newglossaryentry{perplexity}{
  name={Perplejidad},
  description={Métrica que evalúa qué tan bien un modelo de lenguaje predice una muestra de texto},
  first={Perplejidad\glsadd{perplexity}}
}

\newglossaryentry{retriever}{
  name={Recuperador},
  description={Componente de un sistema de recuperación de información que selecciona documentos relevantes a partir de una consulta},
  first={Recuperador\glsadd{retriever}}
}

\newglossaryentry{generator}{
  name={Generador},
  description={Componente de un sistema de recuperación de información que crea respuestas a partir de los documentos recuperados},
  first={Generador\glsadd{generator}}
}

\newglossaryentry{hallucinations}{
  name={Alucinaciones},
  description={Errores en la generación de texto que resultan en respuestas incoherentes o incorrectas},
  first={Alucinaciones\glsadd{hallucinations}}
}

\newglossaryentry{mdp}{
  name={MDP},
  description={Markov Decision Process - Marco matemático para modelar la toma de decisiones en situaciones donde los resultados son parcialmente aleatorios y parcialmente bajo control},
  first={Markov Decision Process (MDP)\glsadd{mdp}}
}

\newglossaryentry{policy}{
  name={Política},
  description={Estrategia que sigue un agente para determinar sus acciones basándose en el estado actual del estudiante},
  first={Política\glsadd{policy}}
}

\newglossaryentry{reward-function}{
  name={Función de Recompensa},
  description={Función que define la retroalimentación que recibe el agente basada en el progreso del estudiante},
  first={Función de Recompensa\glsadd{reward-function}}
}

\newglossaryentry{reward}{
  name={Recompensa},
  description={Valor numérico que indica el éxito o fracaso de una acción tomada por el agente},
  first={Recompensa\glsadd{reward}}
}

\newglossaryentry{engagement}{
  name={Compromiso},
  description={Indicador de la motivación y el interés del estudiante en la tarea de aprendizaje},
  first={Compromiso\glsadd{engagement}}
}

\newglossaryentry{exploitation}{
  name={Explotación},
  description={Estrategia de selección de acciones que se basa en la información ya conocida para maximizar la recompensa},
  first={Explotación\glsadd{exploitation}}
}

\newglossaryentry{exploration}{
  name={Exploración},
  description={Estrategia de selección de acciones que busca descubrir nuevas opciones y mejorar el conocimiento del agente},
  first={Exploración\glsadd{exploration}}
}

\newglossaryentry{knowledge-base}{
  name={Base de Conocimiento},
  description={Conjunto de datos estructurados que almacena información relevante para un sistema de recuperación de información},
  first={Base de Conocimiento\glsadd{knowledge-base}}
}

\newglossaryentry{whisper}{
  name=Whisper,
  description={Modelo de reconocimiento de voz de código abierto desarrollado por OpenAI, capaz de realizar transcripción y traducción en múltiples idiomas},
  first={Whisper\glsadd{whisper}}
}

\newglossaryentry{coqui}{
  name=Coqui TTS,
  description={Suite de herramientas open source para síntesis de voz que incluye múltiples arquitecturas y modelos pre-entrenados},
  first={Coqui TTS\glsadd{coqui}}
}

\newglossaryentry{mfcc}{
  name=MFCC,
  description={Mel-Frequency Cepstral Coefficients. Coeficientes que representan el espectro de potencia a corto plazo de un sonido, basados en una transformación coseno lineal de un espectro de potencia logarítmico en una escala de frecuencia mel no lineal},
  first={Mel-Frequency Cepstral Coefficients (MFCC)\glsadd{mfcc}}
}

\newglossaryentry{viterbi}{
  name=Viterbi,
  description={Algoritmo que encuentra la secuencia más probable de estados ocultos en un modelo oculto de Markov, comúnmente usado en reconocimiento de voz para decodificación},
  first={Viterbi\glsadd{viterbi}}
}

\newglossaryentry{beam-search}{
  name=Beam Search,
  description={Algoritmo de búsqueda heurística que explora un grafo construyendo el grafo gradualmente desde la raíz, expandiendo el nodo más prometedor en un conjunto limitado de nodos},
  first={Beam Search\glsadd{beam-search}}
}

% Path: secciones/material.tex

\newglossaryentry{ml}{
  name={Machine Learning},
  description={Rama de la inteligencia artificial que permite a los sistemas aprender y mejorar a partir de la experiencia},
  first={Machine Learning\glsadd{ml}}
}

% Path: secciones/metodos.tex

\newglossaryentry{ppo}{
  name={PPO},
  description={Proximal Policy Optimization - Algoritmo de aprendizaje por refuerzo que optimiza políticas de control en entornos de decisión continuos y estocásticos},
  first={Proximal Policy Optimization (PPO)\glsadd{ppo}}
}

\newglossaryentry{assistant-ui}{
  name={Assistant UI},
  description={Framework de código abierto para la creación de interfaces de chat conversacionales},
  first={Assistant UI\glsadd{assistant-ui}}
}

\newglossaryentry{chromadb}{
  name={ChromaDB},
  description={Base de datos de código abierto para la gestión de documentos y búsqueda semántica},
  first={ChromaDB\glsadd{chromadb}}
}

\newglossaryentry{cefr}{
  name={CEFR},
  description={Common European Framework of Reference for Languages - Marco Común Europeo de Referencia para las Lenguas},
  first={Common European Framework of Reference for Languages (CEFR)\glsadd{cefr}}
}

\newglossaryentry{api-rest}{
  name={API REST},
  description={Interfaz de programación de aplicaciones basada en el protocolo HTTP y los métodos de petición GET, POST, PUT y DELETE},
  first={API REST\glsadd{api-rest}}
}

\newglossaryentry{escala-likert}{
  name={Escala de Likert},
  description={Método de medición psicométrica que evalúa actitudes y opiniones mediante una escala de respuesta con un rango de opciones},
  first={Escala de Likert\glsadd{escala-likert}}
}

% Acrónimos
\newacronym{bert}{BERT}{Bidirectional Encoder Representations from Transformers}
\newacronym{gan}{GAN}{Generative Adversarial Network}



