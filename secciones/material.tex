\chapter{Material}
\label{material}

Este capítulo detalla los recursos tecnológicos, infraestructura y herramientas utilizadas en el desarrollo del sistema de aprendizaje de idiomas. Se describe la arquitectura general, los componentes hardware y software, así como las bibliotecas y frameworks empleados.

\section{Infraestructura y Recursos Computacionales}

El sistema se implementa localmente utilizando una estación de trabajo de alto rendimiento, aprovechando las capacidades de aceleración por hardware para el procesamiento de modelos de lenguaje y voz.

\subsection{Recursos Hardware}

\begin{itemize}
	\item \textbf{GPU}: NVIDIA GeForce RTX 4070 con las siguientes características:
	      \begin{itemize}
		      \item 12GB de memoria VRAM GDDR6X
		      \item Soporte para CUDA y Tensor Cores
		      \item Capacidades de aceleración para \gls{ml} y \gls{ia}
	      \end{itemize}

	\item \textbf{Memoria Principal}:
	      \begin{itemize}
		      \item 32GB de RAM DDR4
		      \item Optimizada para cargas de trabajo intensivas en memoria
	      \end{itemize}

	\item \textbf{Almacenamiento}:
	      \begin{itemize}
		      \item 1TB SSD NVMe
		      \item Alto rendimiento en lectura/escritura
		      \item Almacenamiento de modelos y datos
	      \end{itemize}
\end{itemize}


\section{Componentes del Sistema}

El sistema se ha diseñado siguiendo una arquitectura modular y escalable que integra tecnologías de vanguardia en \gls{ia} y procesamiento de lenguaje natural. La arquitectura se divide en dos componentes principales: frontend y backend, comunicados a través de una \gls{api-rest}.

\subsection{Backend}

\begin{itemize}
	\item \textbf{LangChain}: Una herramienta poderosa para:
	      \begin{itemize}
		      \item Integrar modelos de lenguaje de gran escala (\gls{llm}) en el sistema
		      \item Gestionar y optimizar prompts para mejorar la interacción con los modelos de lenguaje
		      \item Procesar y analizar texto de manera eficiente utilizando técnicas avanzadas de procesamiento de lenguaje natural
		      \item Posibilita tener acceso a /gls{rag} para mejorar la precisión y relevancia de las respuestas generadas
	      \end{itemize}

	\item \textbf{FastAPI}: Un framework robusto para la creación de servicios de backend y la exposición de APIs, permitiendo una comunicación eficiente con el frontend:
	      \begin{itemize}
		      \item APIs REST de alto rendimiento y baja latencia
		      \item Generación automática de documentación interactiva mediante OpenAPI
		      \item Validación automática de datos y serialización eficiente
	      \end{itemize}
\end{itemize}

\subsubsection{Procesamiento de Voz}
\begin{itemize}
	\item \textbf{Faster-Whisper} (\cite{peng2024owsmv31betterfaster}): Motor de reconocimiento de voz que proporciona:
	      \begin{itemize}
		      \item Transcripción de audio a texto de alta precisión
		      \item Soporte multilingüe robusto
		      \item Optimización para CPU y GPU
	      \end{itemize}

	\item \textbf{Kokoro-TTS} (\cite{hexgrad_2025}): Sistema de síntesis de voz que ofrece:
	      \begin{itemize}
		      \item Generación de voz natural y expresiva
		      \item Múltiples voces y estilos
		      \item Alta eficiencia en el procesamiento
	      \end{itemize}
\end{itemize}

\subsubsection{Modelos de Lenguaje de Gran Escala (\gls{llm})}

\begin{itemize}
	\item \textbf{Phi-4 de Microsoft} (\cite{abdin2024phi4technicalreport}): Modelo avanzado de 14 mil millones de parámetros que impulsa las capacidades conversacionales del sistema:
		  \begin{itemize}
			  \item \textbf{Arquitectura y Entrenamiento}: Construido sobre una combinación estratégica de conjuntos de datos sintéticos, contenido web filtrado de dominio público, y recursos académicos especializados.
			  \item \textbf{Capacidad de Contexto}: 16,000 tokens, permitiendo mantener conversaciones extensas y retener información contextual relevante para el aprendizaje de idiomas.
			  \item \textbf{Ventajas para el Sistema}:
					\begin{itemize}
						\item Operación eficiente en entornos con restricciones computacionales.
						\item Baja latencia en interacciones conversacionales, crucial para la fluidez en el aprendizaje de idiomas.
						\item Capacidades avanzadas de razonamiento para análisis lingüísticos y correcciones gramaticales precisas.
						\item Generación de respuestas contextualmente apropiadas en múltiples idiomas.
					\end{itemize}
			  \item \textbf{Implementación en el Sistema}: El modelo se utiliza para:
					\begin{itemize}
						\item Generar diálogos educativos adaptados al nivel \gls{cefr} del estudiante.
						\item Analizar errores gramaticales y proporcionar retroalimentación pedagógica.
						\item Simular conversaciones auténticas en escenarios prácticos.
						\item Adaptar dinámicamente el contenido a las necesidades específicas del usuario.
					\end{itemize}
		  \end{itemize}

	\item \textbf{Nomic Embed} \cite{nussbaum2024nomic}: Modelo de embeddings de texto de alto rendimiento:
		\begin{itemize}
			\item \textbf{Características Principales}:
				\begin{itemize}
				  \item Ventana de contexto extendida de 8192 tokens
				  \item Modelo de código abierto bajo licencia Apache-2
				  \item Entrenamiento transparente con datos y código disponibles
				  \item Superior a OpenAI Ada-002 en benchmarks de contexto corto y largo
				\end{itemize}
			\item \textbf{Aplicación en el Sistema}:
				\begin{itemize}
				  \item Generación de embeddings para búsqueda semántica
				  \item Soporte a funcionalidades de RAG
				  \item Representación vectorial de conceptos lingüísticos
				  \item Análisis contextual de textos educativos
				\end{itemize}
		\end{itemize}
\end{itemize}

\section{Bases de Datos}

\begin{itemize}
	\item \textbf{Base de Datos SQL}: Almacenamiento de:
		  \begin{itemize}
			  \item Perfiles de usuarios: Información personal y preferencias de los usuarios.
			  \item Progreso de aprendizaje: Registro detallado del avance y desempeño de los usuarios en las actividades de aprendizaje.
			  \item Métricas de rendimiento: Datos estadísticos sobre el uso del sistema y la efectividad de las actividades de aprendizaje.
		  \end{itemize}

	\item \textbf{ChromaDB}: Base de datos vectorial para:
		  \begin{itemize}
			  \item Almacenamiento de embeddings: Representaciones vectoriales de datos textuales y de voz para facilitar la búsqueda y análisis.
			  \item Búsqueda semántica: Capacidad de realizar consultas basadas en el significado y contexto de los datos, en lugar de palabras clave exactas.
		      \item Recuperación de contexto: Extracción de información relevante y contextual para mejorar la interacción y respuestas del sistema.
	      \end{itemize}

	\item \textbf{Redis}: Sistema de caché en memoria para:
	      \begin{itemize}
		      \item Gestión de sesiones de usuario
		      \item Caché de respuestas frecuentes
		      \item Almacenamiento temporal de estados
	      \end{itemize}
\end{itemize}

\subsection{Frontend}

\begin{itemize}
	\item \textbf{Next.js}: Framework de React que ofrece:
	      \begin{itemize}
		      \item Renderizado híbrido (SSR y CSR): Permite la generación de contenido tanto en el servidor como en el cliente, mejorando el rendimiento y la experiencia del usuario.
		      \item Optimización automática de recursos: Gestión eficiente de imágenes, scripts y estilos para mejorar la velocidad de carga.
		      \item Soporte para API Routes: Facilita la creación de endpoints API directamente en la aplicación Next.js.
	      \end{itemize}


	\item \textbf{NextAuth.js}: Sistema de autenticación que proporciona:
	      \begin{itemize}
		      \item Múltiples proveedores de autenticación (OAuth, credenciales)
		      \item Gestión de sesiones segura
		      \item Integración con middleware de Next.js
	      \end{itemize}


	\item \textbf{Next-i18next}: Sistema de internacionalización que ofrece:
	      \begin{itemize}
		      \item Soporte para múltiples idiomas
		      \item Detección automática del idioma del navegador
		      \item Traducciones en el servidor y cliente
	      \end{itemize}
\end{itemize}

\section{Recursos Lingüísticos}

\subsection{Modelos de Voz}
\begin{itemize}
	\item \textbf{Síntesis de Voz (\gls{tts})}:
	      \begin{itemize}
		      \item Generación de voz natural y fluida mediante Kokoro-TTS
		      \item Soporte para 8 idiomas principales:
		            \begin{itemize}
			            \item Inglés (en)
			            \item Español (es)
			            \item Francés (fr)
			            \item Hindi (hi)
			            \item Italiano (it)
			            \item Portugués (pt)
			            \item Japonés (ja)
			            \item Chino (zh)
		            \end{itemize}
		      \item Personalización de voces y estilos de habla
		      \item Optimización para diferentes contextos conversacionales
	      \end{itemize}

	\item \textbf{Reconocimiento de Voz (\gls{stt})}:
	      \begin{itemize}
		      \item Transcripción precisa mediante Faster-Whisper
		      \item Soporte extendido para 20 idiomas:
		            \begin{itemize}
			            \item Lenguas germánicas: Inglés, Alemán, Holandés, Danés, Sueco
			            \item Lenguas románicas: Español, Francés, Italiano, Portugués, Rumano
			            \item Lenguas eslavas: Checo, Polaco, Ruso, Ucraniano
			            \item Lenguas asiáticas: Hindi, Japonés, Coreano, Chino
			            \item Otras lenguas: Árabe, Turco
		            \end{itemize}
		      \item Procesamiento optimizado para CPU y GPU
		      \item Alta precisión en diversos acentos y dialectos
	      \end{itemize}
\end{itemize}


\subsection{Recursos Educativos}
\begin{itemize}
	\item \textbf{Material Didáctico CEFR}:
	      \begin{itemize}
		      \item Contenidos alineados con niveles A1 a C2 del Marco Común Europeo
		      \item Progresión gradual y estructurada del aprendizaje
		      \item Generación sintética de frases adaptadas al nivel \gls{cefr}:
		            \begin{itemize}
			            \item Vocabulario controlado por nivel
			            \item Estructuras gramaticales graduadas
			            \item Complejidad léxica adaptativa
		            \end{itemize}
	      \end{itemize}

	\item \textbf{Escenarios de Práctica}:
	      \begin{itemize}
		      \item Situaciones comunes predefinidas para role-play:
		            \begin{itemize}
			            \item Encuentros sociales básicos
			            \item Transacciones comerciales
			            \item Situaciones profesionales
			            \item Contextos académicos
			            \item Emergencias y asistencia
		            \end{itemize}
		      \item Ejercicios interactivos graduados:
		            \begin{itemize}
			            \item Comprensión lectora y auditiva
			            \item Producción oral y escrita
			            \item Retroalimentación personalizada en tiempo real
		            \end{itemize}
		      \item Práctica contextualizada:
		            \begin{itemize}
			            \item Escenarios de la vida real
			            \item Diálogos situacionales
			            \item Simulaciones de conversaciones auténticas
		            \end{itemize}
	      \end{itemize}
\end{itemize}
