\chapter{Estado del arte}
\label{estado-del-arte}

\section{Sistemas de Aprendizaje Adaptativo}
Los sistemas modernos de aprendizaje de idiomas han evolucionado significativamente con la integración de \gls{ia} y aprendizaje automático. A continuación, se destacan algunas innovaciones recientes:

\begin{itemize}
  \item \textbf{Busuu Conversations (2024)}\footnote{\url{https://www.busuu.com}}: Incorpora un sistema de \gls{ia} que analiza patrones de error y ajusta dinámicamente el contenido para mejorar la eficacia del aprendizaje.
  \item \textbf{Duolingo Max (2024)}\footnote{\url{https://www.duolingo.com}}: Utiliza GPT-4 para generar explicaciones personalizadas y mantener conversaciones contextuales, adaptándose al nivel del usuario.
  \item \textbf{Babbel Everyday Conversations (2023)}\footnote{\url{https://www.babbel.com}}: Combina \gls{ia} con tutores humanos para optimizar la experiencia de aprendizaje híbrido, ofreciendo una interacción más personalizada.
  \item \textbf{Lingvist (2023)}\footnote{\url{https://www.lingvist.com}}: utiliza datos contextuales para generar ejercicios, lecciones y recomendaciones adaptadas, facilitando la recuperación de contenidos lingüísticos relevantes y la generación de actividades interactivas.
\end{itemize}

\section{Aplicaciones de LLM}
Los \gls{llm} han revolucionado el aprendizaje de idiomas en múltiples aspectos, proporcionando herramientas avanzadas para la generación de diálogos y el análisis y corrección de textos.

\subsection{Asistentes y Diálogo}
\begin{itemize}
  \item \textbf{ChatGPT (2022)}\footnote{\url{https://chatgpt.com/}}: Revolucionó la interacción humano-IA estableciendo el estándar de interfaces conversacionales naturales y creando un ecosistema completo de desarrollo.
  \item \textbf{Claude (2023)}\footnote{\url{https://claude.ai/}}: Destacó por su precisión superior en análisis de documentos y capacidad de seguir instrucciones complejas con menor tendencia a la alucinación.
  \item \textbf{Azure Language Studio (2023)}\footnote{\url{https://language.cognitive.azure.com/}}: Ofrece herramientas de análisis lingüístico y generación de contenido educativo, mejorando la calidad del aprendizaje.
  \item \textbf{LLaMA (2023)}\footnote{\url{https://ai.facebook.com/blog/large-language-model-llama}}: Modelo de \gls{open-source} desarrollado por Meta, diseñado para ser eficiente y accesible para la investigación y aplicaciones prácticas.
\end{itemize}

\subsection{Análisis y Corrección}

\begin{itemize}
  \item \textbf{Grammarly with GrammarlyGO (2023)}\footnote{\url{https://www.grammarly.com}}: Utiliza \gls{ia} generativa para proporcionar correcciones contextuales y sugerencias de mejora, ayudando a los usuarios a escribir con mayor precisión.
  \item \textbf{DeepL Write (2023)}\footnote{\url{https://www.deepl.com/write}}: Sistema de corrección que considera el contexto cultural y el registro lingüístico, ofreciendo sugerencias más relevantes y precisas.
\end{itemize}

\section{Tecnologías Emergentes con Compañeros de Aprendizaje}

Los compañeros de aprendizaje basados en \gls{ia} están emergiendo como una herramienta fundamental en la educación moderna. Estos asistentes virtuales proporcionan apoyo personalizado y adaptativo, actuando como tutores disponibles 24/7. A continuación, se presentan algunas implementaciones destacadas:

\begin{itemize}
  \item \textbf{Khanmigo (2024)}\footnote{\url{https://www.khanacademy.org/khan-labs}}: Tutor virtual de Khan Academy que actúa como compañero de estudio personalizado, proporcionando explicaciones adaptativas, guía paso a paso y retroalimentación instantánea en múltiples materias.

  \item \textbf{Third Space Learning (2024)}\footnote{\url{https://thirdspacelearning.com}}: Plataforma que combina tutores humanos con \gls{ia} para crear una experiencia de aprendizaje híbrida, donde el sistema analiza las interacciones y proporciona insights personalizados.

  \item \textbf{Riiid SANTA (2023)}\footnote{\url{https://riiid.com}}: Sistema de tutoría adaptativa para predecir el rendimiento del estudiante y personalizar el contenido, maximizando la eficiencia del aprendizaje mediante análisis predictivo.
\end{itemize}

\section{Avances en Procesamiento de Voz}

Las tecnologías de \gls{tts} y \gls{stt} han mejorado en naturalidad y expresividad, proporcionando voces más humanas y adaptativas, así como una transcripción precisa y rápida. A continuación, se presentan algunas de las tecnologías más destacadas en este campo:

\begin{itemize}
  \item \textbf{Whisper OpenAI (2022)}\footnote{\url{https://openai.com/research/whisper}}: Reconocimiento de voz multilingüe de alta precisión, eficaz en ambientes ruidosos y con diversos acentos. Es \gls{open-source} y se utiliza para transcripción automática y análisis de voz en múltiples idiomas.
  \item \textbf{Google Speech-to-Text/Text-to-Speech (2023)}\footnote{\url{https://cloud.google.com/speech-to-text}}: Reconocimiento de voz en tiempo real con alta precisión, soporte para múltiples idiomas y fácil integración con otras plataformas de Google. Comúnmente usado en asistentes virtuales y transcripción de reuniones en vivo.
  \item \textbf{Microsoft Azure AI Speech (2023)}\footnote{\url{https://azure.microsoft.com/en-us/products/ai-services/ai-speech}}: Transcripción precisa y rápida, con capacidades avanzadas de personalización y adaptación al contexto. Ideal para sistemas de atención al cliente y análisis de conversaciones en tiempo real.
  \item \textbf{Deepgram (2023)}\footnote{\url{https://deepgram.com}}: Plataforma de reconocimiento de voz basada en redes neuronales profundas, conocida por su rapidez y precisión. Utilizada para transcripción de llamadas y análisis de conversaciones de negocio.
\end{itemize}

\section{Agentic AI}

La tecnología de \gls{multi-agent} se esta convirtiendo en un área clave de innovación en el aprendizaje de idiomas. Estas tecnologías permiten la creación de agentes autónomos que pueden interactuar entre sí y con los usuarios para proporcionar experiencias de aprendizaje más dinámicas y personalizadas.

\begin{itemize}
  \item \textbf{LangChain (2022)}\footnote{\url{https://www.langchain.com}}: Plataforma \gls{open-source} que facilita la creación de \gls{multi-agent}. LangChain permite la integración de diferentes modelos de lenguaje y agentes especializados para tareas específicas, mejorando la interacción y la adaptabilidad del sistema.
  \item \textbf{CrewAI (2023)}\footnote{\url{https://www.crewai.com}}: Sistema multi-agente \gls{open-source} diseñado para la colaboración en equipo, permitiendo a los usuarios trabajar juntos en proyectos de aprendizaje de idiomas y recibir retroalimentación en tiempo real.
  \item \textbf{phiData (2023)}\footnote{\url{https://www.phidata.com}}: Plataforma \gls{open-source} que utiliza agentes especializados para analizar datos lingüísticos y proporcionar recomendaciones personalizadas para mejorar el aprendizaje de idiomas.
  \item \textbf{Autogen de Microsoft (2023)}\footnote{\url{https://www.microsoft.com/en-us/research/project/autogen}}: Tecnología \gls{open-source} de Microsoft que permite la creación de agentes autónomos para tareas específicas en el aprendizaje de idiomas, mejorando la personalización y la eficacia del proceso educativo.
\end{itemize}

\section{Frameworks de Aprendizaje por Refuerzo}

El \gls{rl} ha ganado popularidad en la industria debido a su capacidad para resolver problemas complejos mediante la optimización de políticas a través de la interacción con el entorno. A continuación, se presentan algunos de los frameworks de \gls{rl} más utilizados en la industria, todos ellos \gls{open-source}:

\begin{itemize}
  \item \textbf{TensorFlow Agents (2019)}\footnote{\url{https://www.tensorflow.org/agents}}: Una biblioteca de \gls{rl} basada en \gls{tensorflow} que proporciona herramientas para construir, entrenar y evaluar agentes de \gls{rl}. Es compatible con una amplia gama de algoritmos y entornos.
  \item   \item \textbf{Stable Baselines3 (2020)}\footnote{\url{https://stable-baselines3.readthedocs.io}}: Una implementación de algoritmos de \gls{rl} en \gls{py-torch}, diseñada para ser fácil de usar y extender. Es ampliamente utilizada para experimentación y desarrollo de soluciones de \gls{rl}.
  \item \textbf{TorchRL (2022)}\footnote{\url{https://github.com/pytorch/rl}}: Un framework de aprendizaje por refuerzo basado en \gls{py-torch}, diseñado para ser flexible y fácil de usar. Proporciona herramientas para construir, entrenar y evaluar agentes de \gls{rl} en diversos entornos.
\end{itemize}
