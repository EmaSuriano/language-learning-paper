\chapter{Estado del arte}
\label{estado-del-arte}

Esta sección presenta una revisión detallada de las tecnologías y sistemas más avanzados en el campo del aprendizaje de idiomas asistido por \gls{ia}. El objetivo es contextualizar la presente investigación dentro del panorama actual, identificando tendencias, avances significativos y oportunidades de innovación. Se analizan seis áreas tecnológicas clave: sistemas de aprendizaje adaptativo, aplicaciones de \gls{llm}, compañeros de aprendizaje basados en IA, avances en procesamiento de voz, sistemas multi-agente y frameworks de \gls{rl}. Finalmente, se explica cómo estas tecnologías se integran en la propuesta de este trabajo.

\section{Sistemas de Aprendizaje Adaptativo}
\label{sec:sistemas-aprendizaje-adaptativo}


Los sistemas modernos de aprendizaje de idiomas han evolucionado significativamente en los últimos años, incorporando algoritmos avanzados de \gls{ia} y aprendizaje automático que permiten una adaptación precisa al nivel y necesidades de cada usuario. Esta evolución representa un cambio de paradigma frente a los enfoques tradicionales estáticos, permitiendo experiencias educativas personalizadas y dinámicas \cite{roll2018learning}.

A continuación, se analizan las plataformas más innovadoras del mercado, destacando sus principales características tecnológicas y enfoques pedagógicos:


\begin{itemize}
  \item \textbf{Busuu Conversations (2024)}\footnote{\url{https://www.busuu.com}}: Incorpora un sistema de \gls{ia} que analiza patrones de error y ajusta dinámicamente el contenido para mejorar la eficacia del aprendizaje.
  \item \textbf{Duolingo Max (2024)}\footnote{\url{https://www.duolingo.com}}: Utiliza GPT-4 para generar explicaciones personalizadas y mantener conversaciones contextuales, adaptándose al nivel del usuario.
  \item \textbf{Babbel Everyday Conversations (2023)}\footnote{\url{https://www.babbel.com}}: Combina \gls{ia} con tutores humanos para optimizar la experiencia de aprendizaje híbrido, ofreciendo una interacción más personalizada.
  \item \textbf{Lingvist (2023)}\footnote{\url{https://www.lingvist.com}}: utiliza datos contextuales para generar ejercicios, lecciones y recomendaciones adaptadas, facilitando la recuperación de contenidos lingüísticos relevantes y la generación de actividades interactivas.
  \item \textbf{Elsa Speak (2023)}\footnote{\url{https://www.elsaspeak.com}}: Sistema de pronunciación asistida por \gls{ia} que proporciona retroalimentación en tiempo real y ejercicios personalizados para mejorar la fluidez y la precisión en la pronunciación.
\end{itemize}

Estos sistemas representan la vanguardia actual en aprendizaje adaptativo de idiomas, pero como señala \cite{vanlehn2011relative}, aún existen desafíos significativos en cuanto a la modelización precisa del conocimiento del estudiante y la adaptación a estilos de aprendizaje diversos, áreas donde la presente investigación busca contribuir.

\section{Aplicaciones de LLM en Educación Lingüística}
\label{sec:aplicaciones-llm}

Los \gls{llm} han transformado radicalmente el panorama del aprendizaje de idiomas, proporcionando capacidades sin precedentes para la generación de diálogos naturales, análisis contextual y corrección inteligente. En esta sección se analizan las principales aplicaciones de estas tecnologías, categorizadas en sistemas conversacionales y herramientas de análisis textual.


\subsection{Asistentes y Diálogo}
\label{subsec:asistentes-dialogo}

La evolución de los sistemas conversacionales ha alcanzado un nivel de sofisticación que permite interacciones casi humanas, ofreciendo entornos de práctica lingüística altamente efectivos:


\begin{itemize}
  \item \textbf{ChatGPT (2022)}\footnote{\url{https://chatgpt.com/}}: Revolucionó la interacción humano-IA estableciendo el estándar de interfaces conversacionales naturales y creando un ecosistema completo de desarrollo.
  \item \textbf{Claude (2023)}\footnote{\url{https://claude.ai/}}: Destacó por su precisión superior en análisis de documentos y capacidad de seguir instrucciones complejas con menor tendencia a la alucinación.
  \item \textbf{Azure Language Studio (2023)}\footnote{\url{https://language.cognitive.azure.com/}}: Ofrece herramientas de análisis lingüístico y generación de contenido educativo, mejorando la calidad del aprendizaje.
  \item \textbf{LLaMA (2023)}\footnote{\url{https://ai.facebook.com/blog/large-language-model-llama}}: Modelo de \gls{open-source} desarrollado por Meta, diseñado para ser eficiente y accesible para la investigación y aplicaciones prácticas.
\end{itemize}

\subsection{Análisis y Corrección}
\label{subsec:analisis-correccion}

Las herramientas de análisis y corrección basadas en \gls{llm} han evolucionado más allá de la simple identificación de errores gramaticales, incorporando comprensión contextual profunda y recomendaciones estilísticas:


\begin{itemize}
  \item \textbf{Grammarly with GrammarlyGO (2023)}\footnote{\url{https://www.grammarly.com}}: Utiliza \gls{ia} generativa para proporcionar correcciones contextuales y sugerencias de mejora, ayudando a los usuarios a escribir con mayor precisión.
  \item \textbf{DeepL Write (2023)}\footnote{\url{https://www.deepl.com/write}}: Sistema de corrección que considera el contexto cultural y el registro lingüístico, ofreciendo sugerencias más relevantes y precisas.
\end{itemize}

El avance de estos sistemas, sin embargo, plantea desafíos importantes relacionados con la dependencia excesiva de la corrección automatizada y el potencial impacto en la autonomía del aprendizaje \cite{Rodriguez2023}, aspectos que deben considerarse cuidadosamente en el desarrollo de nuevos sistemas educativos basados en \gls{llm}.


\section{Tecnologías Emergentes con Compañeros de Aprendizaje}
\label{sec:companeros-aprendizaje}

Los compañeros de aprendizaje basados en \gls{ia} representan una evolución significativa en los sistemas educativos, implementando una dimensión social y emocional que complementa la transmisión de conocimientos técnicos. Estos sistemas van más allá de la simple instrucción, estableciendo una relación pedagógica que incluye motivación, adaptación personalizada y apoyo constante \citep{baker2014educational}.

\begin{itemize}
  \item \textbf{Khanmigo (2024)}\footnote{\url{https://www.khanacademy.org/khan-labs}}: Tutor virtual de Khan Academy que actúa como compañero de estudio personalizado, proporcionando explicaciones adaptativas, guía paso a paso y retroalimentación instantánea en múltiples materias.

  \item \textbf{Third Space Learning (2024)}\footnote{\url{https://thirdspacelearning.com}}: Plataforma que combina tutores humanos con \gls{ia} para crear una experiencia de aprendizaje híbrida, donde el sistema analiza las interacciones y proporciona insights personalizados.

  \item \textbf{Riiid SANTA (2023)}\footnote{\url{https://riiid.com}}: Sistema de tutoría adaptativa para predecir el rendimiento del estudiante y personalizar el contenido, maximizando la eficiencia del aprendizaje mediante análisis predictivo.
\end{itemize}

Estos sistemas de compañeros de aprendizaje representan una dirección prometedora para el futuro de la educación lingüística, ya que proporcionan un entorno de práctica personalizado y adaptativo que puede complementar significativamente los métodos tradicionales.

\section{Avances en Procesamiento de Voz}
\label{sec:procesamiento-voz}

Las tecnologías de procesamiento de voz, incluyendo \gls{tts} y \gls{stt}, han experimentado avances revolucionarios en los últimos años, transformando radicalmente las posibilidades para el aprendizaje de pronunciación y comprensión auditiva. Estos sistemas han evolucionado desde voces robóticas y reconocimiento limitado hasta alcanzar niveles de naturalidad y precisión casi humanos \citep{graves2013speech}.


\begin{itemize}
  \item \textbf{Whisper OpenAI (2022)}\footnote{\url{https://openai.com/research/whisper}}: Reconocimiento de voz multilingüe de alta precisión, eficaz en ambientes ruidosos y con diversos acentos. Es \gls{open-source} y se utiliza para transcripción automática y análisis de voz en múltiples idiomas.
  \item \textbf{Google Speech-to-Text/Text-to-Speech (2023)}\footnote{\url{https://cloud.google.com/speech-to-text}}: Reconocimiento de voz en tiempo real con alta precisión, soporte para múltiples idiomas y fácil integración con otras plataformas de Google. Comúnmente usado en asistentes virtuales y transcripción de reuniones en vivo.
  \item \textbf{Microsoft Azure AI Speech (2023)}\footnote{\url{https://azure.microsoft.com/en-us/products/ai-services/ai-speech}}: Transcripción precisa y rápida, con capacidades avanzadas de personalización y adaptación al contexto. Ideal para sistemas de atención al cliente y análisis de conversaciones en tiempo real.
  \item \textbf{Deepgram (2023)}\footnote{\url{https://deepgram.com}}: Plataforma de reconocimiento de voz basada en redes neuronales profundas, conocida por su rapidez y precisión. Utilizada para transcripción de llamadas y análisis de conversaciones de negocio.
  \item \textbf{Kokoro-82M (2025)}\footnote{\url{https://huggingface.co/hexgrad/Kokoro-82M}}: Kokoro es un modelo TTS de código abierto con 82 millones de parámetros. A pesar de su arquitectura ligera, ofrece una calidad comparable a modelos más grandes, siendo significativamente más rápido y rentable.
\end{itemize}

Estos avances en procesamiento de voz abren nuevas posibilidades para la creación de entornos inmersivos de práctica conversacional, donde los estudiantes pueden desarrollar habilidades comunicativas en contextos realistas con retroalimentación instantánea y personalizada.

\section{Agentic AI}
\label{sec:agentic-ai}

La tecnología de \gls{multi-agent} se está convirtiendo en un área clave de innovación en el aprendizaje de idiomas. Estas tecnologías permiten la creación de agentes autónomos que pueden interactuar entre sí y con los usuarios para proporcionar experiencias de aprendizaje más dinámicas y personalizadas. El enfoque multi-agente supera las limitaciones de los sistemas monolíticos al distribuir responsabilidades entre agentes con roles específicos, mejorando tanto la eficacia como la robustez del sistema \cite{Liu2023}.

\begin{itemize}
  \item \textbf{LangChain (2022)}\footnote{\url{https://www.langchain.com}}: Plataforma \gls{open-source} que facilita la creación de \gls{multi-agent}. LangChain permite la integración de diferentes modelos de lenguaje y agentes especializados para tareas específicas, mejorando la interacción y la adaptabilidad del sistema.
  \item \textbf{CrewAI (2023)}\footnote{\url{https://www.crewai.com}}: Sistema multi-agente \gls{open-source} diseñado para la colaboración en equipo, permitiendo a los usuarios trabajar juntos en proyectos de aprendizaje de idiomas y recibir retroalimentación en tiempo real.
  \item \textbf{phiData (2023)}\footnote{\url{https://www.phidata.com}}: Plataforma \gls{open-source} que utiliza agentes especializados para analizar datos lingüísticos y proporcionar recomendaciones personalizadas para mejorar el aprendizaje de idiomas.
  \item \textbf{Autogen de Microsoft (2023)}\footnote{\url{https://www.microsoft.com/en-us/research/project/autogen}}: Tecnología \gls{open-source} de Microsoft que permite la creación de agentes autónomos para tareas específicas en el aprendizaje de idiomas, mejorando la personalización y la eficacia del proceso educativo.
\end{itemize}

El paradigma de agentes autónomos representa una dirección prometedora para el desarrollo de sistemas educativos de próxima generación, permitiendo crear ecosistemas adaptativos que simulan los complejos roles pedagógicos que tradicionalmente han sido exclusivos de los instructores humanos.

\section{Frameworks de Aprendizaje por Refuerzo}
\label{sec:frameworks-rl}

El \gls{rl} ha demostrado ser un paradigma particularmente adecuado para el desarrollo de sistemas educativos adaptativos, gracias a su capacidad intrínseca para optimizar estrategias a través de interacciones secuenciales, similar al proceso natural de aprendizaje humano \cite{williams2017educational}. Los frameworks modernos de \gls{rl} proporcionan herramientas robustas para implementar estos sistemas a escala.


\begin{itemize}
  \item \textbf{TensorFlow Agents (2019)}\footnote{\url{https://www.tensorflow.org/agents}}: Una biblioteca de \gls{rl} basada en \gls{tensorflow} que proporciona herramientas para construir, entrenar y evaluar agentes de \gls{rl}. Es compatible con una amplia gama de algoritmos y entornos.
  \item \textbf{Stable Baselines3 (2020)}\footnote{\url{https://stable-baselines3.readthedocs.io}}: Una implementación de algoritmos de \gls{rl} en \gls{py-torch}, diseñada para ser fácil de usar y extender. Es ampliamente utilizada para experimentación y desarrollo de soluciones de \gls{rl}.
  \item \textbf{TorchRL (2022)}\footnote{\url{https://github.com/pytorch/rl}}: Un framework de aprendizaje por refuerzo basado en \gls{py-torch}, diseñado para ser flexible y fácil de usar. Proporciona herramientas para construir, entrenar y evaluar agentes de \gls{rl} en diversos entornos.
\end{itemize}

La elección de Stable Baselines3 para la implementación del sistema propuesto en este trabajo se fundamenta en su equilibrio óptimo entre facilidad de uso y flexibilidad, así como en su robusta implementación del algoritmo \gls{ppo}, que ha demostrado ser particularmente efectivo para problemas de optimización de secuencias educativas \citep{schulman2017proximal}.


\section{Aplicación de las Tecnologías en el Trabajo}
\label{sec:aplicacion-tecnologias}

Este trabajo integra las tecnologías más avanzadas identificadas en el estado del arte para desarrollar un sistema comprehensivo de aprendizaje de idiomas. La propuesta sintetiza múltiples enfoques tecnológicos en una arquitectura cohesiva y sinérgica, donde cada componente aporta capacidades específicas al sistema global.

\begin{itemize}
  \item \textbf{Sistemas de Aprendizaje Adaptativo}: Se implementa un sistema que analiza los patrones de error de los usuarios y ajusta dinámicamente el contenido. Este enfoque se inspira en las capacidades adaptativas de Busuu Conversations, pero incorpora un modelado multidimensional del conocimiento que considera interdependencias entre diferentes habilidades lingüísticas.
  
  \item \textbf{Aplicaciones de LLM}: Se utiliza un \gls{llm} para generar diálogos y proporcionar correcciones contextuales. La integración de Phi-4 optimizado específicamente para el contexto educativo combina capacidades conversacionales naturales con precisión en la evaluación y retroalimentación lingüística.
  
  \item \textbf{Tecnologías Emergentes con Compañeros de Aprendizaje}: Se desarrolla un asistente virtual que actúa como compañero de aprendizaje, proporcionando apoyo personalizado y adaptativo. Inspirado en la arquitectura de Khanmigo, implementa estrategias de scaffolding basadas en el nivel actual y estilo de aprendizaje detectado del estudiante.
  
  \item \textbf{Avances en Procesamiento de Voz}: Se integra tecnología de \gls{tts} y \gls{stt} para mejorar la interacción del usuario con el sistema. La implementación combina Faster-Whisper para reconocimiento de voz y Kokoro-TTS para síntesis, proporcionando una experiencia de comunicación oral naturalista y precisa.
  
  \item \textbf{Agentic AI}: Se explora la creación de agentes autónomos que interactúan entre sí y con los usuarios. El sistema implementa una arquitectura multi-agente basada en LangChain, donde agentes especializados colaboran en diferentes aspectos del proceso educativo: tutoría, evaluación, motivación y práctica conversacional.
  
  \item \textbf{Frameworks de Aprendizaje por Refuerzo}: Se utilizan frameworks de \gls{rl} para optimizar el proceso de aprendizaje y adaptar el contenido a las necesidades de los usuarios. Específicamente, se implementa el algoritmo \gls{ppo} mediante Stable Baselines3 para optimizar las rutas de aprendizaje y la adaptación dinámica de niveles.
\end{itemize}

Estas tecnologías y teorías se integran en un sistema unificado que supera las limitaciones de enfoques fragmentados, proporcionando una experiencia de aprendizaje de idiomas que es adaptativa, interactiva y altamente personalizada, mejorando significativamente tanto la eficacia educativa como la experiencia del usuario.

Este análisis del estado del arte proporciona la base contextual para comprender cómo nuestra propuesta se sitúa en el panorama actual de tecnologías para el aprendizaje de idiomas. En el siguiente capítulo, se profundizará en el marco teórico que fundamenta los diversos componentes del sistema, estableciendo los principios pedagógicos y computacionales que guían su diseño.