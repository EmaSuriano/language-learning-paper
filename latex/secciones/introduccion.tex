\chapter{Introducción}
\label{introduccion}

El aprendizaje de idiomas en la era digital ha experimentado una transformación significativa gracias a los avances en \gls{ia}. Sin embargo, uno de los mayores desafíos sigue siendo la personalización efectiva del proceso de aprendizaje para adaptarse a las necesidades individuales de cada estudiante. Este trabajo propone un enfoque innovador que combina técnicas de \gls{rl} con arquitecturas \gls{transformers} introducidas y tecnologías de procesamiento de voz para crear un sistema de aprendizaje de idiomas adaptativo y personalizado.

\citep{vaswani2017attention} introdujo la arquitectura \gls{transformers} que ha revolucionado el procesamiento del lenguaje natural. \citep{vaswani2017attention} demostraron que los \gls{transformers} superan a las arquitecturas recurrentes y convolucionales en tareas de traducción automática. \citep{devlin2018bert} presentaron \gls{bert}, un modelo de lenguaje basado en \gls{transformers} que ha establecido nuevos estándares en tareas de procesamiento del lenguaje natural.

\section{Motivación}
\label{motivacion}

La adquisición de una segunda lengua es un proceso complejo que varía significativamente entre individuos. Los métodos tradicionales de enseñanza de idiomas, incluso en su forma digitalizada, presentan limitaciones significativas.

\subsection{Limitaciones Actuales}
\label{limitaciones-actuales}

\begin{itemize}
  \item \textbf{Rigidez Estructural}: Los programas siguen secuencias predefinidas que no se adaptan al progreso real del estudiante.
  \item \textbf{Falta de Personalización}: No consideran adecuadamente los diferentes estilos de aprendizaje y preferencias individuales.
  \item \textbf{Retroalimentación Limitada}: La mayoría de los sistemas proporcionan feedback básico sin considerar el contexto completo del aprendizaje.
  \item \textbf{Práctica Conversacional Artificial}: Las interacciones suelen ser mecánicas y no reflejan la naturaleza dinámica del lenguaje real.
\end{itemize}

\subsection{Oportunidades de Mejora}
\label{oportunidad-de-mejora}

\begin{itemize}
  \item \textbf{Adaptabilidad Dinámica}: Sistemas que ajustan el contenido y la dificultad en tiempo real.
  \item \textbf{Personalización Profunda}: Consideración de múltiples factores individuales para optimizar el aprendizaje.
  \item \textbf{Interacción Natural}: Uso de tecnologías avanzadas para simular conversaciones más realistas.
  \item \textbf{Feedback Contextual}: Retroalimentación detallada y específica basada en el perfil del estudiante.
\end{itemize}

\section{Objetivos}
\label{objetivos}

\subsection{Objetivo General}
\label{objetivo-general}

Desarrollar un sistema de aprendizaje de idiomas que utilice \gls{rl}, \gls{transformers} y una arquitectura multi-agente para una experiencia de aprendizaje personalizada, adaptativa y efectiva.

\subsection{Objetivo Específicos}
\label{objetivos-especificos}

\begin{enumerate}
  \item \textbf{Optimización del Aprendizaje}
        \begin{itemize}
          \item Implementar un sistema de \gls{rl} que optimice rutas de aprendizaje personalizadas
          \item Desarrollar mecanismos de adaptación dinámica del contenido
          \item Crear sistemas de evaluación continua del progreso
        \end{itemize}

  \item \textbf{Mejora de la Interacción}
        \begin{itemize}
          \item Integrar modelos \gls{transformers} para procesamiento del lenguaje natural
          \item Desarrollar sistemas de diálogo contextuales
          \item Implementar análisis de errores en tiempo real
        \end{itemize}

  \item \textbf{Perfeccionamiento de Habilidades Lingüísticas}
        \begin{itemize}
          \item Crear sistemas de evaluación de pronunciación usando \gls{tts} y \gls{stt}
          \item Desarrollar ejercicios adaptativos de comprensión
          \item Implementar práctica conversacional contextual
        \end{itemize}

  \item \textbf{Gestión del Conocimiento}
        \begin{itemize}
          \item Integrar sistemas \gls{rag} para acceso a recursos educativos
          \item Desarrollar bases de conocimiento dinámicas
          \item Implementar mecanismos de actualización de contenido
        \end{itemize}
\end{enumerate}