\chapter{Estado del arte}
\label{estado-del-arte}

\section{Sistemas de Aprendizaje Adaptativo}
Los sistemas modernos de aprendizaje de idiomas han evolucionado significativamente con la integración de \gls{ia} y aprendizaje automático. Destacan las siguientes innovaciones:

\begin{itemize}
  \item \textbf{Busuu (2023)} incorpora un sistema de IA que analiza patrones de error y ajusta dinámicamente el contenido.

  \item \textbf{Duolingo Max (2023)} utiliza GPT-4 para generar explicaciones personalizadas y mantener conversaciones contextuales.

  \item \textbf{Babbel Live (2023)} combina IA con tutores humanos para optimizar la experiencia de aprendizaje híbrido.
\end{itemize}

\section{Aplicaciones de LLM}
Los \gls{llm} han revolucionado el aprendizaje de idiomas en múltiples aspectos:

\subsection{Generación de Diálogos}
\begin{itemize}
  \item \textbf{ChatGPT for Language Learning (2023)}: Capacidad de mantener conversaciones multilingües con adaptación de nivel.

  \item \textbf{LangChain Applications (2023)}: Framework para crear agentes conversacionales especializados en enseñanza de idiomas.

  \item \textbf{Microsoft Azure Language Studio (2023)}: Herramientas de análisis lingüístico y generación de contenido educativo.
\end{itemize}

\subsection{Análisis y Corrección}
\begin{itemize}
  \item \textbf{Grammarly with GrammarlyGO (2023)}: Utiliza \gls{ia} generativa para proporcionar correcciones contextuales y sugerencias de mejora.

  \item \textbf{DeepL Write (2023)}: Sistema de corrección que considera el contexto cultural y el registro lingüístico.
\end{itemize}

\section{Tecnologías Emergentes en Educación Lingüística}

\section{Sistemas Multi-Agente}
Las implementaciones recientes destacan por:

\begin{itemize}
  \item \textbf{Microsoft AI Tutor (2023)}: Sistema multi-agente que combina diferentes roles pedagógicos.

  \item \textbf{OpenAI GPTs (2023)}: Agentes especializados para diferentes aspectos del aprendizaje de idiomas.

  \item \textbf{Anthropic Claude (2023)}: Capacidad de mantener contexto extenso y proporcionar explicaciones detalladas.
\end{itemize}

\section{RAG en Educación}
Los sistemas RAG han mostrado resultados prometedores:

\begin{itemize}
  \item \textbf{Lingua RAG (2023)}: Sistema que combina conocimiento lingüístico estructurado con generación de contenido.

  \item \textbf{EduRAG Framework (2023)}: Arquitectura para recuperación y generación de contenido educativo personalizado.
\end{itemize}

\section{Avances en Procesamiento de Voz}

\section{Tecnologías de Síntesis y Reconocimiento}
\begin{itemize}
  \item \textbf{Whisper OpenAI (2023)}: Sistema de reconocimiento de voz multilingüe de alta precisión.

  \item \textbf{Azure Neural TTS (2023)}: Voces naturales con control de emotividad y estilo.

  \item \textbf{Google Cloud Speech Services (2023)}: API unificada para reconocimiento y síntesis multilingüe.
\end{itemize}

\section{Integración de Tecnologías}

\section{Sistemas Híbridos}
Las soluciones más avanzadas combinan múltiples tecnologías:

\begin{itemize}
  \item \textbf{AI Language Coach (2023)}: Integra \gls{llm}, \gls{stt}/\gls{tts} y análisis de progreso.

  \item \textbf{Adaptive Learning Platforms (2023)}: Combinan \gls{rl} con sistemas multi-agente para optimizar rutas de aprendizaje.

  \item \textbf{Immersive Language Learning (2023)}: Une realidad virtual con procesamiento de lenguaje natural.
\end{itemize}

\section{Tendencias Futuras}

\todo{Agregar referencia del trabajo}
Las direcciones más prometedoras incluyen:

\begin{itemize}
  \item \textbf{Personalización Profunda}: Sistemas que adaptan no solo el contenido sino también la metodología de enseñanza.

  \item \textbf{Aprendizaje Multimodal}: Integración de diferentes modalidades de input y output.

  \item \textbf{Sistemas Autónomos}: Agentes que pueden mantener conversaciones extensas y naturales.

  \item \textbf{Evaluación Continua}: Sistemas que ajustan el aprendizaje en tiempo real basado en múltiples métricas.
\end{itemize}