\chapter{Marco teórico}
\label{marco-teorico}

\section{Reinforcement Learning}

\subsection{Fundamentos}

\todo{15 a 20 paginas}
\todo{agregar una introducción a este capítulo}
\todo{Agregar referencia bibliográfica a cada sección}

El RL se basa en el paradigma de aprendizaje a través de la interacción con un entorno. Los componentes principales son:

\begin{equation}
  Q(s,a) \leftarrow Q(s,a) + \alpha[r + \gamma \max_{a'}Q(s',a') - Q(s,a)]
\end{equation}

Donde:

\section{Arquitecturas Transformer}
\subsection{Mecanismo de Atención}
El mecanismo de atención se puede expresar matemáticamente como:

\begin{equation}
  \text{Attention}(Q,K,V) = \text{softmax}\left(\frac{QK^T}{\sqrt{d_k}}\right)V
\end{equation}

\section{Sistema Multi-Agente con RAG}
\subsection{Arquitectura de Agentes}
La arquitectura propuesta integra múltiples agentes especializados:

\begin{itemize}
  \item \textbf{Agente Tutor}
        \begin{itemize}
          \item Gestión del proceso de aprendizaje
          \item Coordinación con otros agentes
          \item Seguimiento del progreso
        \end{itemize}

  \item \textbf{Agente de Contenido}
        \begin{itemize}
          \item Recuperación de material
          \item Generación de ejercicios
          \item Adaptación de dificultad
        \end{itemize}
\end{itemize}