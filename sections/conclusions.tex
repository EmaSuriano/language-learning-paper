\chapter{Conclusions}
\label{conclusions}

This chapter presents the conclusions derived from the development and implementation of the language learning system based on \gls{rl} techniques and \gls{transformers} architectures. It analyzes the achievements, contributions made, limitations identified, and future lines of research and development.

\section{Project Achievements}
\label{sec:project-achievements}

The present work has succeeded in developing a comprehensive language learning system that meets the initially set objectives:

\begin{itemize}
    \item \textbf{Adaptive personalization}: A system based on \gls{ppo} has been successfully implemented that optimizes the learning path according to the individual profile and progress of the student, dynamically adjusting the difficulty of the content.
    
    \item \textbf{Advanced conversational interaction}: The integration of \gls{llm} models and a \gls{rag} system has enabled the generation of contextualized and natural dialogues, providing a realistic conversational experience.
    
    \item \textbf{Development of comprehensive language skills}: The system successfully integrates voice processing technologies (\gls{tts} and \gls{stt}) for the simultaneous development of listening comprehension and oral production skills.
    
    \item \textbf{Efficient knowledge management}: The implementation of the \gls{rag} system provides contextualized access to relevant educational resources, improving the accuracy and relevance of the system's responses.
    
    \item \textbf{Modular and extensible architecture}: The system design allows for its evolution and adaptation to new requirements, facilitating the incorporation of improvements and new functionalities.
\end{itemize}

\section{Contributions}
\label{sec:contributions}

The main contributions of this work to the field of AI-assisted language learning are:

\subsection{Technical Advances}
\label{subsec:technical-advances}

\begin{itemize}
    \item \textbf{PPO model optimized for education}: A \gls{ppo} model specifically adapted to the educational context has been developed, capable of making informed pedagogical decisions based on multiple performance metrics.
    
    \item \textbf{Effective integration of LLM and RAG}: The system demonstrates an efficient implementation of the combination of large language models with retrieval-augmented generation, providing contextually relevant and educationally significant responses.
    
    \item \textbf{Optimized voice processing pipeline}: The adaptation of Faster-Whisper and Kokoro-TTS for the educational context represents a significant optimization in terms of efficiency and accuracy for language learning applications.
\end{itemize}

\subsection{Methodological Contributions}
\label{subsec:methodological-contributions}

\begin{itemize}
    \item \textbf{Multidimensional evaluation framework}: A systematic approach has been developed to evaluate both the technical performance of the system and its real educational impact.
    
    \item \textbf{Generation of representative scenarios}: The methodology developed for creating and evaluating representative learning scenarios provides a useful framework for future research in adaptive systems.
    
    \item \textbf{Student-centered design}: The project has implemented an approach that prioritizes the student experience, adapting technologies to real pedagogical needs.
\end{itemize}

\section{Limitations of the Work}
\label{sec:limitations}

Despite the achievements, it is important to recognize the current limitations of the system:

\begin{itemize}
    \item \textbf{Preliminary evaluation}: The tests conducted, although promising, have been limited to a small group of users in a controlled environment. More extensive and longitudinal studies are required to fully validate the effectiveness of the system.
    
    \item \textbf{Linguistic coverage}: Although the system supports multiple languages, the quality and depth of the educational resources vary significantly among them, with greater robustness in major languages such as English and Spanish.
    
    \item \textbf{Computational resource dependency}: The current system requires considerable computational resources, which may limit its accessibility in environments with technological constraints.
    
    \item \textbf{Cultural aspects of language}: The system shows limitations in understanding and generating culturally specific aspects of language, such as idioms, humor, or local cultural references.
\end{itemize}

\section{Future Lines}
\label{sec:future-lines}

This work opens various lines of research and future development:

\subsection{Short-term Technical Improvements}
\label{subsec:short-term-improvements}

\begin{itemize}
    \item \textbf{Expansion of the knowledge base}: Expand and enrich the knowledge base of the \gls{rag} system, incorporating more diverse and updated educational resources.
    
    \item \textbf{Improvement of the correction system}: Implement more sophisticated techniques for the detection and correction of linguistic errors in real-time.
\end{itemize}

\subsection{Long-term Vision}
\label{subsec:long-term-vision}

\begin{itemize}
    \item \textbf{Multimodal systems}: Integrate multimodal understanding and generation (text, voice, gestures, facial expressions) for a more immersive and complete learning experience.
    
    \item \textbf{Adaptation to specific contexts}: Develop specialized versions of the system for specific educational contexts, such as language teaching for specific purposes (tourism, business, medicine, etc.).
    
    \item \textbf{Collaborative learning}: Explore the integration of collaborative learning systems that encourage interaction between students and the co-creation of knowledge.
\end{itemize}

\section{Final Reflections}
\label{sec:final-reflections}

The development of this system represents a significant step toward the effective personalization of language learning through \gls{ai} technologies. The combination of \gls{rl}, \gls{transformers} architectures, and \gls{rag} systems demonstrates the potential of current technologies to fundamentally transform the educational field.

The main value of the system does not lie solely in its technical capabilities, but in its potential to democratize access to personalized and effective learning experiences. Dynamic adaptation to individual needs allows for overcoming the limitations of traditional approaches, which often fail to provide the specific support that each student requires.

However, it is important to recognize that technology, no matter how advanced, represents only a tool at the service of broader pedagogical objectives. The developed system does not aim to replace human educators, but to complement their work, providing an environment for constant practice and feedback that enriches the overall educational experience.

The true measure of this project's success will be its ability to facilitate language learning in a more efficient, inclusive, and motivating way, thus helping to break down the linguistic barriers that separate people and communities in an increasingly interconnected world.

As a final reflection, it is worth highlighting that the field of \gls{ai} applied to education is constantly evolving, and this work represents only a starting point for future developments that will continue to transform the way we learn and teach languages.